I would like to express my deepest gratitude to my advisor, Dr. Ajit Joglekar. Pursuing a Ph.D. degree is arduous, especially for me as an international student during a challenging time. Ajit has always been supporting me in many aspects. % To clearly convey the idea to the audience today, what I presented earlier was an over-simplified version of the story. For a very long time, we have got results as in previous studies published by 2 different groups that the recruitment of BUBR1 by BUB1 to signaling KTs do not seem to contribute to the SAC. In fact, it is known that BUBR1 additionally will bring in PP2A-B56 and reduce the SAC sigaling activity. Ajit always believes that we can control the expression level of BUBR1 and cancel the effect of PP2A-B56. And after over 2 years of modifying the conditions, we finally showed that the recruitment of BUBR1 by BUB1 per se contributes to the SAC.

I am grateful to all members of my thesis committee, who have kindly offered their time and expertise to give me guidance, suggestions, and comments on my research for nearly six years.

I would like to acknowledge all of my lab colleagues (Dr. Arti Dumbrepatil, Martin Fernandez, Frank Ferrari, Adrienne Fontan, Simon Han, Lauren Humphrey-Stark, Soubhagyalaxmi Jema, Shriya Karmarkar, Sisira Kavuri, Dr. Alex Kukreja, Andrew Livingston, Sean McGuire, Anna Olson, Juan Orozco, Rebekah Ronan, Dr. Babhrubahan Roy, Janice Sim, Dr. Vikash Verma, and Alex Vita) who have supported my work in the lab by directly participating in my research (which is credited in corresponding parts of the main text of this thesis), offering valuable suggestions, and building up a healthy work environment.

Our collaborators have taught me a lot about science and have greatly enhanced the breadth and impact of our research. They include (but are not limited to): Dr. Iain Cheeseman (Whitehead Institute); Dr. John Tyson and Dr. Anand Banerjee (Virginia Tech); Dr. Geert Kops (Hubrecht Institute, Netherlands); Dr. Song-Tao Liu and Dr. Yibo Luo (University of Toledo); Dr. Andrea Musacchio and Dr. Valentina Piano (MPI of Molecular Physiology, Germany).

Many other friends and colleagues have helped me greatly in this endeavor, either by directly solving my technical problems or offering key suggestions over the years. They include (but are not limited to): Dr. Mara Duncan (experimental techniques); Dr. J. Damon Hoff (fluorescence spectroscopy and associated data analysis); Dr. Dawei Yang (image processing and pattern recognition); Dr. Uhn-Soo Cho as well as Dr. Sojin An and Dr. Jennifer Chik (molecular cloning and protein purification); Dr. Diane Fingar and Dr. Dubek Kazyken (kinase assay); Dr. Fengrong Wang and Dr. Takamasa Inoue (immunoprecipitation); Dr. Chengxin Zhang (structure prediction); Dr. Xiangtian Tan (bioinformatics); Dr. Nigel Michki and Dr. Kyoung Jo (fluorescence imaging); Dr. Zhejian Ji (spindle assembly checkpoint). Many others provided me with necessary reagents and instruments, whose assistance was or will be acknowledged in my published paper \cite{eSAC} and other manuscripts submitted to peer-reviewed journals \cite{KImotifPaper, Paper3} or in preparation.

I would like to thank my mentors during my first-year rotations (Dr. Tom Kerppola and Dr. Wei Cheng). Although I did not join their labs, I am thankful for their close guidance at the early stage of my academic pursuit.%Also, I have broadened my view through exchanges with many members from Dr. Ryoma Ohi's lab and Dr. Uhn-Soo Cho's lab.

Last but not least, I could not have reached so far without the support and inspiration from my family and friends. They are my pole star and lighthouses in this memorable voyage.