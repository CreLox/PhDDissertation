The equal distribution of replicated genetic materials packed within chromosomes into daughter cells is the hallmark of mitosis. Failure to maintain such genome stability is often associated with cancers and many pathological syndromes. It is safeguarded by various surveillance pathways, including the spindle assembly checkpoint (SAC) which my thesis revolves around. The segregation of chromosomes is mainly carried out by the spindle, wherein microtubules attach to chromosomes through kinetochores. The SAC is activated to block the progression of mitosis at kinetochores lacking end-on microtubule attachment.

The biochemical events associated with the SAC have been mostly elucidated. It is activated primarily by the phosphorylation of the consensus MELT motifs on the scaffold protein KNL1 by the kinase MPS1 at signaling kinetochores. However, how the SAC effectively blocks the progression of mitosis in the presence of just a single unattached kinetochore remains unclear. In my thesis research, I studied how multiple proteins and reactions cooperate at multiple layers to enable such sensitivity of the SAC.

To first confirm if positive cooperativity plays a role in the SAC, we need a quantitative readout of the input (how many MELT motifs are phosphorylated) and the output (the duration of the mitotic arrest). Gauging the phosphorylation of MELT motifs on endogenous KNL1 proteins in live cells is difficult. Therefore, in Chapter 2, we engineered a cytosolic probe that ectopically activates the SAC, which solved this technical challenge. Dose-response analyses revealed that under certain conditions, the SAC signaling activity was stronger with a smaller number of phosphorylated KNL1 proteins. This striking observation strongly indicated positive cooperativity in the SAC.

Next, in Chapter 3, we sought to validate some of the essential preconditions and important implications from our cooperativity model in Chapter 2 in the kinetochore-based SAC. It is known that some BUBR1 proteins are recruited to signaling kinetochores together with other SAC proteins that make up the mitotic checkpoint complex (MCC, which is the effector that blocks the progression of mitosis). We showed that the recruitment of BUBR1 \textit{per se} contributed to the activity of the kinetochore-based SAC signaling. We also demonstrated that the cytosolic pools of SAC proteins were effectively diminished due to their recruitment to signaling kinetochores and that the numbers of SAC proteins recruited per signaling kinetochore were inversely correlated with the total number of signaling kinetochores in the cell.

In Chapter 4, we explored the catalytic mechanism of the formation of the CDC20-MAD2 dimer (an MCC subcomplex). Because both CDC20 and MAD2 bind to MAD1, we hypothesized that the structural flexibility of MAD1 helps to position CDC20 and MAD2 in proximity and maneuvers them, thereby coordinating MAD2's conformational switch with the CDC20-MAD2 dimerization. Our data showed that disrupting the structural flexibility of MAD1 impaired the SAC signaling activity, which shed light on this cooperativity scaffolded by MAD1.

Future studies are needed to achieve a complete understanding of how cooperativity contributes to the sensitivity of the SAC.