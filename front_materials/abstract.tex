Equal distribution of replicated genetic materials packed within chromosomes into daughter cells is the hallmark of mitosis. This task is mainly carried out by the spindle, whose microtubules attach to chromosomes through adaptors named kinetochores. Failure to achieve faithful chromosome segregation is often associated with cancers and many pathological syndromes. It is safeguarded by the spindle assembly checkpoint (SAC) which my thesis revolves around. The SAC is activated to block the progression of mitosis at kinetochores lacking end-on microtubule attachment.

There are 92 unattached kinetochores at the beginning of mitosis in a canonical somatic human cell. The SAC can prevent premature chromosome segregation in the presence of either 92 or just a single unattached kinetochore. Although the biochemical events associated with the SAC have been mostly elucidated, how the SAC reaches its sensitivity to robustly block the progression of mitosis in the presence of a single unattached kinetochore remains unclear. In my thesis research, I studied how multiple proteins and reactions cooperate at multiple layers to enable such sensitivity.

The SAC is activated primarily by the phosphorylation of the consensus MELT motifs on the scaffold protein KNL1 at signaling kinetochores. To study the origin of the sensitivity of the SAC, we need a quantitative readout of the input (how many MELT motifs are phosphorylated) and the output (the duration of the mitotic arrest caused by the SAC). In Chapter 2, we engineered a cytosolic probe that ectopically activates the SAC, which solved this technical challenge to gauge the phosphorylation of MELT motifs in live cells. Dose-response analyses revealed that under certain conditions, the SAC signaling activity was stronger with a smaller number of phosphorylated KNL1 proteins. This striking observation strongly indicated positive cooperativity in the SAC, which could underlie the sensitivity of the SAC.

Next, in Chapter 3, we sought to validate some of the preconditions of this cooperativity model in the context of the endogenous kinetochore-based SAC. We demonstrated that the cytosolic pools of SAC proteins were effectively diminished due to their recruitment to signaling kinetochores and that the numbers of SAC proteins recruited per signaling kinetochore were inversely correlated with the total number of signaling kinetochores in the cell. Additionally, we showed for the first time that the recruitment of BUBR1, an important constituent of the mitotic checkpoint complex (MCC, the effector molecule that blocks the progression of mitosis), contributed to the activity of the kinetochore-based SAC signaling. These observations reinforced the idea that cooperative signaling among MELT motifs in proximity forms the basis of the sensitivity of the SAC.

In Chapter 4, we explored the catalytic mechanism of the formation of CDC20-MAD2, which is considered the rate-limiting step in the assembly of MCC. Both CDC20 and MAD2 bind to MAD1, so we hypothesized that the structural flexibility of MAD1 helps to position CDC20 and MAD2 closely, thereby coordinating the formation of CDC20-MAD2. Our data showed that disrupting the structural flexibility of MAD1 impaired the SAC signaling activity, based on which we proposed a cooperativity model for the catalytic mechanism of the formation of CDC20-MAD2 scaffolded by MAD1.

My thesis research strongly implies the role of cooperativity in shaping the sensitivity of the SAC. However, future studies are needed to fill in the gaps toward a more complete understanding of the origin of the sensitivity of the SAC.