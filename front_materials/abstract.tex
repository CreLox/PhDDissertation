Equal distribution of replicated genetic materials packed in chromosomes from a parent cell to two daughter cells is the hallmark of mitosis. This task is carried out by the spindle, whose microtubules attach to chromosomes through adaptors named kinetochores. Failure to achieve faithful chromosome segregation is often associated with cancers and many pathological syndromes.

My thesis investigates a surveillance mechanism that safeguards faithful chromosome segregation named the spindle assembly checkpoint (SAC), which is activated at kinetochores lacking end-on spindle microtubule attachment to block the progression of mitosis. The SAC can robustly arrest a mammalian cell in mitosis in the presence of either many unattached kinetochores or merely one. Although the biochemical events associated with the SAC have been mostly elucidated, how the SAC reaches such ``sensitivity'' to effectively arrest a cell in mitosis in the presence of a single unattached kinetochore remains unclear. In my thesis research, I studied how multiple proteins and reactions cooperate at various layers to enable this sensitivity.

First, to study the origin of the sensitivity of the SAC, we need a quantitative readout of the input (the phosphorylation of the scaffold protein KNL1 which recruits SAC proteins and activates the SAC at signaling kinetochores) and the output (the duration of the mitotic arrest caused by the SAC). In Chapter 2, we engineered a cytosolic probe that ectopically activates the SAC, which solved the technical challenge to gauge the phosphorylation of KNL1 in live cells. Dose-response analyses revealed that under certain conditions, the SAC signaling activity was stronger with a smaller number of phosphorylated KNL1 proteins. This striking observation strongly indicated positive cooperativity in the SAC, which could underlie the sensitivity of the SAC.

Next, I sought to validate some of the preconditions of this cooperativity model above in the context of the endogenous kinetochore-based SAC in Chapter 3. We demonstrated %that the cytosolic pools of SAC proteins were diminished due to their recruitment to signaling kinetochores and
that the numbers of SAC proteins recruited per signaling kinetochore were inversely correlated with the total number of signaling kinetochores in the cell. Additionally, I showed for the first time that the localization at signaling kinetochores of BUBR1, an important constituent of the mitotic checkpoint complex (MCC, the effector that blocks the progression of mitosis), strengthened the SAC activity. These observations reinforced the idea that cooperative signaling contributes to the sensitivity of the SAC.

Finally, in Chapter 4, I explored the catalytic mechanism of the formation of CDC20-MAD2, which is considered the rate-limiting step in the assembly of MCC. Understanding this catalysis is fundamental to comprehending how an unattached kinetochore may produce an effective signal to stall mitosis. Given that both CDC20 and MAD2 bind to the catalyst MAD1, I hypothesized that the structural flexibility of MAD1 helps to position CDC20 and MAD2 closely, thereby coordinating the formation of CDC20-MAD2. Our data showed that disrupting the structural flexibility of MAD1 impaired the SAC signaling activity, based on which I proposed a cooperativity model for the catalytic mechanism of the formation of CDC20-MAD2 scaffolded by MAD1.

My thesis research reveals the role of cooperativity in shaping the sensitivity of the SAC. It also establishes the foundation for future studies aiming for a complete understanding of the origin of its sensitivity.