The equal distribution of replicated genetic materials packed within chromosomes into daughter cells is the hallmark of somatic cell division (mitosis). Failure to maintain chromosome stability is detrimental and often associated with cancers and many pathological syndromes. It is safeguarded by various surveillance pathways, including the spindle assembly checkpoint (SAC) which my thesis revolves around. The segregation of chromosomes is mainly carried out by the spindle, formed by microtubules emanating from two poles of the cell and attaching to chromosomes through kinetochores. The SAC is activated to block the progression of mitosis in the presence of unattached kinetochores.

Through decades of study, the biochemical reactions associated with the SAC have been mostly elucidated. It is activated primarily by the phosphorylation of the consensus MELT motifs on the scaffold protein KNL1 by the kinase MPS1 at signaling kinetochores. However, a fundamental question about the SAC remains unclear: how does the SAC effectively block the progression of cell division in the presence of even just a single unattached kinetochore? Taking a cue from different well-studied biological decision-making processes, I hypothesized that cooperativity boosts the sensitivity of the SAC. In my thesis research, I deployed cell biology, biochemistry, and biophysics approaches to study if and how multiple proteins and reactions cooperate to enable the sensitivity of the SAC.

To confirm if positive cooperativity plays a role in the SAC, we first need a quantitative readout of the input (how many MELT motifs are phosphorylated to activate the SAC) and the output (the duration from the mitotic arrest as a common indicator of the SAC signaling activity). Gauging phosphorylation of MELT motifs on KNL1 at a single cell level is technically difficult. Therefore, in Chapter 2, we engineered a cytosolic probe to ectopically activate the SAC. This probe utilizes chemically-induced dimerization of the kinase domain of MPS1 and the phosphodomain of KNL1. Our dose-response analyses revealed that under certain conditions, the SAC signaling activity is stronger with a smaller number of phosphorylated KNL1 proteins. This striking observation of "less is more" strongly indicates positive cooperativity in the SAC. To our knowledge, this is the first published evidence supporting the existence of cooperativity in the SAC.

We next seek to validate this finding in the kinetochore-based SAC signaling in Chapter 3. We show that the cytosolic pools of SAC proteins were effectively diminished due to their recruitment to signaling kinetochores and that the numbers of SAC proteins recruited per signaling kinetochore are inversely correlated with the total number of signaling kinetochores in the cell. It is known that a fraction of the BUBR1 proteins are recruited to signaling kinetochores together with other SAC proteins that make up the mitotic checkpoint complex (MCC) which is the effector molecule that prevents anaphase onset. We show that the recruitment of \protein{BubR1} \Latin{per se} contributes to the activity of the kinetochore-based SAC signaling. Together, these results validate that the enrichment of SAC proteins at kinetochores strengthens the SAC, which are important implications of the model in Chapter 2.

In Chapter 4, we explored the spatio-temporal coupling between MAD2's conformational switch and the formation of CDC20-MAD2 dimer (an MCC subcomplex), which has been suggested by recent literature. Because both CDC20 and MAD2 bind to MAD1, we hypothesize that the structural flexibility of MAD1 helps to position CDC20 and MAD2 in proximity and maneuver them, thereby coordinating MAD2's conformational switch with the CDC20-MAD2 dimerization. Our data show that disrupting the structural flexibility of MAD1 impairs the SAC signaling activity, which sheds light on this cooperativity scaffolded by MAD1.

Through my thesis research, I show that cooperativity works at multiple layers in the SAC signaling pathway. Future studies are needed to achieve a complete understanding of how cooperativity contributes to the sensitivity of the SAC.