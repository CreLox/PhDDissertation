During cell division, the unequal distribution of replicated genetic materials packed within chromosomes results in abnormal numbers of chromosomes (aneuploidy) in daughter cells, which is often associated with cancers. Multiple key surveillance pathways regulate the cell cycle to ensure genetic stability. My thesis research revolves around one of these pathways -- the spindle assembly checkpoint (SAC). Over the course of attachment, the number of unattached kinetochores decreases from 92 (when all kinetochores are unattached) to 0 (when the attachment is complete) in human cells. The SAC monitors the completeness of the attachment process and activates to block the progession of somatic cell division (mitosis) if it detects unattached kinetochores. A complete understanding of how the SAC safeguards against aneuploidy will teach us how to prevent or induce its failure, which may inspire novel methods to prevent or to treat cancers. The revelation of how different molecules orchestrate to build a de facto logic gate from biochemical reactions in the SAC may also spark innovations in biological circuit design and protein engineering.

Through decades of study, the field has mostly elucidated the biochemical reactions associated with the SAC. However, despite the deep knowledge of the participating proteins and reactions, a fundamental question about the SAC remains unclear. How does the SAC, a biochemical signaling cascade, approaches the sensitivity of a binary electronic logic gate, blocking the progression of cell division in the presence of even just a single unattached kinetochore and releasing the block promptly once all attachment is secured? Taking a cue from different well-studied biological decision-making processes, I hypothesized that cooperativity boosts the sensitivity of the SAC. Just like effective collaborations between people in the society, positive cooperativity in biochemical processes can synergize reactions and amplify the output. This synergy and amplification may be fundamental to the sensitivity of the SAC.

In my thesis research, I deployed cell biology (including engineered probe system, gene editing, and live-cell time-lapse imaging), biochemistry (including immunoprecipitation and in vitro reconstitution of biochemical reactions), and biophysics (including fluorescence lifetime imaging and mathematical modeling) approaches to study if and how multiple proteins and reactions cooperate to enable the sensitivity of the SAC.

To confirm if positive cooperativity plays a role in the SAC, we first need a quantitative readout of the input (how many MELT motifs are phosphorylated to activate the SAC) and the output (the duration of cell division as a common indicator of the MCC assembly rate) with sufficient statistical power. Because of the nanoscale volume of a kinetochore where KNL1 proteins reside, gauging phosphorylation of MELT motifs on KNL1 is technically difficult in live cells. To circumvent this obstacle, we engineered a cytosolic probe to ectopically activate the SAC (Chen et al., 2019). This probe utilizes chemically-induced dimerization of MPS1 and fragments of KNL1 that contain various numbers of MELT motifs. Upon the addition of the dimerization reagent, MPS1 comes into proximity with MELT motifs of KNL1 and phosphorylates them, thereby activating the SAC. The probe system is tagged with fluorescent proteins, allowing us to correlate the duration of cell division with the concentration of the SAC-activating probe at a single-cell resolution in real time by microscopy. Our dose-response analyses surprisingly revealed that under certain conditions the SAC signaling strength is stronger with a smaller number of phosphorylated MELT motifs present. This observation of "less is more" strongly indicates positive cooperativity in the SAC, which together with the limiting abundance of certain SAC protein(s) provides a perfect theory to explain the phenomenon. This theoretical framework projects a nonlinear enrichment of certain SAC proteins in the presence of a small number of unattached kinetochores under the physiological settings, which we verified experimentally. To our knowledge, this is the first published evidence supporting the existence of cooperativity in the SAC.

After confirming the cooperativity in the SAC, we move on to inquire into its potential molecular mechanisms and prospective factors that may affect it. Based on our model, we are investigating (1) the distance between MELT motifs and the presence of auxiliary motifs of KNL1, (2) the parallel recruitment pathway of an MCC component named BUBR1, and (3) a potential spatial and temporal coupling of two reactions involving two other MCC components named CDC20 and MAD2.

We started by varying the distance between two MELT motifs in our aforementioned probe system by incorporating various numbers of tandem repeats of a specified linker. Based on tested cases, variation in the linker length between two MELT motif does not seem to yield much difference in the SAC dose-response. We further tested whether auxiliary motifs of KNL1 facilitate cooperativity in the SAC. In contradiction to previous publications, we discovered that these auxiliary motifs of KNL1 sequester BUBR1 instead of contributing to the SAC. This limits the availability of BUBR1, leading to a decreased SAC response. Together with our first paper, we concluded that the presence and the number of different functional motifs on the KNL1 affect the cooperativity in the SAC.

Next, we looked into whether the parallel recruitment pathway of BUBR1 contributes to the cooperativity in the SAC. Some BUBR1 proteins bind to phosphorylated MELT motifs which also recruit other SAC proteins that make up the MCC. We hypothesized that this pool of BUBR1 comes into proximity with other MCC components, increasing their apparent concentrations and thus facilitating the MCC assembly. We knocked down the endogenous BUBR1 and rescued the cell with either the wild-type BUBR1 or a mutated BUBR1 that can only bind to phosphorylated MELT motifs exclusively. Comparing the SAC response of the wild-type BUBR1 with that of the mutated BUBR1...

Lastly, we are currently exploring whether the potential spatial and temporal coupling between the conformational change of MAD2 and the formation of CDC20-MAD2 dimer boosts the efficiency of CDC20-MAD2 dimerization (and hence the MCC assembly). It is known that mixing CDC20 and MAD2 alone in vitro results in a slow dimerization kinetic. In human cells, CDC20 and MAD2 bind to the same protein scaffold MAD1. We hypothesized that the structural flexibility of MAD1 helps to position CDC20 and MAD2 in proximity and maneuver them, thereby catalyzing the conformational change of MAD2 and coupling this process with the CDC20-MAD2 dimerization. Our preliminary data showed that disrupting the structural flexibility of MAD1 impairs the efficiency of the MCC assembly.