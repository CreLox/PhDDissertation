\chapter{Conclusion and Perspective}
\label{chpt:conclusion}

Through my thesis research, I systematically show that cooperative synergy manifests at multiple layers of the core SAC signaling pathway in human cells, which contributes to its sensitivity (effectively delaying the progression of mitosis in the presence of a small number of unattached or laterally attached kinetochores). The revelation of the critical role of cooperative synergy (combined with the competition for the limited pool of signaling proteins) not only expands our understanding of the SAC but may also inspire efforts to examine whether similar mechanisms apply to other signaling pathways that involve the localization of signaling components to signaling scaffolds (proteins and non-coding RNAs \cite{ScaffoldlncRNA}).

Human \protein{Knl1} possesses 19 putative MELT motifs scattered throughout its middle region \cite{MELTEvolution}. In \myref{chpt:2}, by employing different numbers of MELT motifs in the signaling scaffold in our engineered eSAC activator, we demonstrated the phenomenological synergy among the multiple MELT motifs within a single scaffold \cite{eSAC}. The most striking case was the eSAC activator with six MELT motifs in the signaling scaffold \myref{M6_DoseResponse}, wherein cells with a lower abundance of the eSAC activator may invoke a higher SAC signaling activity. This phenomenon was not apparent when the signaling scaffold incorporated two, three, or four MELT motifs (see \myref{M2_DoseResponse,M3_DoseResponse,M4_DoseResponse}), probably due to the heterogeneity in the sequences of the MELT motifs that determines their differential affinities to \protein{Bub1}-\protein{Bub3} \cite{MELTActivity, eSAC}. We further proved that the variation in the distance between two MELT motifs (in the range of 135--311 residues) minimally affected the SAC signaling activity. In \myref{ProzoneEffectModel}, we proposed a model reminiscent of the hook effect to explain our observations.

The MELT motifs phosphorylated by \protein{Mps1} sequentially recruit other SAC proteins like \protein{Bub1}, \protein{BubR1}, and \protein{Mad1} \cite{Ji2017eLife}. Some of the crucial preconditions in our model proposed in \myref{ProzoneEffectModel} include (1) that the competition among a large number of phosphorylated MELT motifs effectively diminishes the limited pool of freely diffusive SAC proteins in the cytosol and (2) that the co-localization of multiple SAC proteins on the same \protein{Knl1} scaffold superlinearly boosts the SAC signaling activity. In \myref{chpt:3}, we demonstrated a negative correlation between the number of signaling kinetochores in a prometaphase cell and the number of \protein{Bub1}, \protein{BubR1}, and \protein{Mad1} proteins recruited per signaling kinetochore. This observation was not due to potential variation in the phosphorylation of \protein{Knl1} at signaling kinetochores. We further showed for the first time that recruitment of \protein{BubR1} by \protein{Bub1} \Latin{per se} contributes to the activity of the kinetochore-based SAC signaling. Although we have not proved the aforementioned superlinearity in kinetochore-based SAC signaling, these pieces of experimental evidence nonetheless unveiled that the foundation of the phenomenological synergy observed in \myref{chpt:2} was (at least partially) the competition among a large number of phosphorylated MELT motifs for the limited pool of downstream SAC proteins, which promote cooperative SAC signaling when they co-localize in close spatial proximity.

The formation of the \protein{Cdc20}-\protein{Mad2} subcomplex is considered the rate-limiting step in the assembly of the MCC, the effector molecule of the SAC. Recent studies implied that \protein{Cdc20} binds to \protein{Bub1} and \protein{Mad1} cooperatively and that \protein{Mad2}'s conformational switch and the formation of the \protein{Cdc20}-\protein{Mad2} dimer may be spatio-temporally coupled \cite{BUB1-CDC20-MAD1, Tripartite}. We independently scrutinized whether and how \protein{Mad1} may scaffold such coupling using mutational studies. We discovered that the structural flexibility of \protein{Mad1} enabled by its loop region was critical to the SAC signaling activity, which supported the notion of such coupling. Based on our experimental data, we proposed a ``knitting'' model to explain the catalytic mechanism of \protein{Mad1} in promoting the formation of the \protein{Cdc20}-\protein{Mad2} dimer. First, \protein{Mad1} in its fold-back conformation physically positions the MIM of \protein{Cdc20} and \protein{Mad2} closely, facilitating their association. Second, the cooperative interaction between the scaffold protein \protein{Mad1} in its fold-back conformation and \protein{Cdc20}-\protein{Mad2} favors the formation of the \protein{Mad1}RWD-\protein{Cdc20}-\protein{Mad2}-\protein{Mad1}MIM complex. Third, the extended conformation breaks such avidity binding and promotes the release of \protein{Cdc20}-\protein{Mad2} into the cytosol. However, future studies are needed to fill in the gaps between our current experimental evidence and the ``knitting'' model.

It should be recognized that my thesis research has some inherent limitations. First, as addressed in \myref{Sensitivity+Responsiveness,eSACDiscussions}, the sensitivity and responsiveness of the SAC are two sides of the same coin. I only focused on the sensitivity of the SAC by activating the SAC in human cells via the eSAC activator or mitotic drugs. However, during normal mitosis, cells usually do not have to rely on the SAC and delay the progression of mitosis for several hours. Future studies should examine these analyses on the sensitivity of the SAC in a more physiological context, with full consideration of the responsiveness of the SAC in human cells \cite{YeastMELTSpecificity}.

Second, my studies revolved around the assembly of the MCC. However, even in the presence of signaling kinetochores, there is a constant tug-of-war between the SAC signaling activity and the counter-acting SAC silencing activity. The silencing of the SAC is mainly mediated by (1) the dephosphorylation of the MELT motifs by phosphatases as well as the shedding of the corona (which together cease the assembly of the MCC) \cite{PP2A-B56, DyneinStripsCorona}, (2) the disassembly or degradation of existing MCCs \cite{BubR1MitosisTurnover, CCT-MCCDisassembly, Ubiquitylation-MCCDisassembly, UBR5-MCCDisassembly, TRIP13-p31-MAD2}, and (3) the attenuation of the affinity between the MCC and the APC/C \cite{APC-SUMO}. Whether the regulations on the mechanisms that silence the SAC also contribute to the sensitivity of the SAC deserves a systematic evaluation \cite{0thOrder, ZeroOrder}.

Finally, mammalian kinetochores usually bind to more than one microtubule in the metaphase and there are two spindle poles from which microtubules may emanate. Therefore, even end-on kinetochore-microtubule attachment could be erroneous. For example, both kinetochores on a duplicated chromosome may attach to microtubules emanating from the same spindle pole (syntelic attachment), and a single kinetochore may attach to microtubules emanating from both spindle poles (merotelic attachment). Cells deploy the error-correction mechanism to correct these erroneous attachment modes to ensure accurate chromosome segregation \cite{Syntelic+Merotelic}. How the sensitivity of the SAC fits into the interplay between the SAC and the error-correction mechanism is a developing topic that may broaden our understanding of the SAC in the physiological context.

% local regulation of cohesin cleavage, MCC, APC/C inhibition.

% Hypothesis-driven top-down approach vs exploratory omics-based bottom-up approach

A more thorough understanding of the SAC may teach us how to prevent or induce its failure and inspire drug discovery to prevent or treat cancers. The revelation of how different molecules orchestrate to build a \Latin{de facto} logic gate from biochemical reactions in the case of the SAC may also spark innovations in biochemical circuit engineering.