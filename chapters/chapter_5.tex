\chapter{Conclusion}
\label{chpt:conclusion}

% Through my studies described in previous chapters, I showed that cooperativity manifest at multiple layers of the core SAC signaling cascade in human cells, which contributes to its sensitivity to effectively delay the progression of mitosis in the presence of even just a small number of unattached or laterally attached kinetochores.

% Human \protein{Knl1} possesses 19 putative MELT motifs scattered throughout its middle region. By employing different numbers of MELT motifs in the signaling scaffold in our engineered eSAC activator in \myref{chpt:2}, we showed that there is phenomenological cooperativity when there are multiple MELT motifs within a single scaffold. The most striking case is when there are six MELT motifs (M3-M3) in the signaling scaffold, wherein cells a smaller number of the eSAC activator may invoke a higher SAC signaling acivity compared to cells with a higher number of the eSAC activator. This phenomenon was not apparent when the signaling scaffold incorporated two, three, or four MELT motifs (see \myref{M2_DoseResponse,M3_DoseResponse,M4_DoseResponse}), probably due to the heterogeneity in the sequences of the MELT motifs and their differential affinity to \protein{Bub1}-\protein{Bub3} \cite{MELTActivity}. In light of variable distances between MELT motifs in human \protein{Knl1}, we further proved that such variation in the distance between two MELT motifs (in the range of 135--311 residues) minimally affected the SAC signaling activity. In \myref{ProzoneEffectModel}, we proposed a model reminiscent of the hook effect in an agglutination test to explain our observations.

% MELT motifs phosphorylated by \protein{Mps1} sequentially recruit other SAC proteins like \protein{Bub1}, \protein{BubR1}, and \protein{Mad1} \cite{Ji2017eLife}. Some of the crucial preconditions in our cooperativity model proposed in Chapter 2 include (1) that the competition among a large number of phosphorylated MELT motifs for the limited pool of SAC proteins effectively diminishes freely diffusive SAC proteins in the cytosol and (2) that the co-localization of multiple SAC proteins on the same \protein{Knl1} scaffold boosts the SAC signaling activity. With the help of genome editing, we showed that the number of there was a negative correlation between the number of signaling kinetochores in a prometaphase cell and the number of \protein{Bub1}, \protein{BubR1}, and \protein{Mad1} proteins recruited per signaling kinetochore. This observation was not due to potential variation in the degree of phosphorylation of \protein{Knl1} at signaling kinetochores. We further showed for the first time that recruitment of \protein{BubR1} by \protein{Bub1} \Latin{per se} contributes to the activity of the kinetochore-based SAC signaling. These pieces of experimental evidence unveiled that the foundation of the phenomenological cooperativity observed in \myref{chpt:2} was the competition among a large number of phosphorylated MELT motifs for the limited pool of downstream SAC proteins, which may promote cooperative SAC signaling when they co-localize in close spatial proximity.

% Recent studies showed that \protein{Cdc20} binds to \protein{Bub1} and \protein{Mad1} cooperatively and that \protein{Mad2}'s conformational switch and the formation of the \protein{Cdc20}-\protein{Mad2} dimer may be spatio-temporally coupled \cite{BUB1-CDC20-MAD1, Tripartite}. We independently looked into whether and how \protein{Mad1} may scaffold such coupling using mutational studies. We discovered that the structural flexibility of \protein{Mad1} enabled by its loop region was critical to the SAC signaling activity, which supported the notion of such coupling. Based on our experimental data, we proposed a ``knitting'' model to explain the catalytic mechanism of \protein{Mad1} in promoting the formation of the \protein{Cdc20}-\protein{Mad2} dimer, which is considered the rate limiting step in the assembly of the MCC. \protein{Mad1} in its fold-back conformation physically positions the MIM of \protein{Cdc20} and \protein{Mad2} in close proximity, providing the possibility for their association. The cooperative interaction between the scaffold protein \protein{Mad1} in its fold-back conformation and the final product (the \protein{Cdc20}-\protein{Mad2} dimer) favors the formation of the \protein{Mad1}RWD-\protein{Cdc20}-\protein{Mad2}-\protein{Mad1}MIM complex. Whether the extended conformation breaks such avidity binding and promotes the release of the \protein{Cdc20}-\protein{Mad2} dimer into the cytosol in a ATP-dependent manner requires future studies.

% How the sensitivity of SAC interplays with respnsiveness?

% (1) The silencing of the SAC is mainly mediated by (1) the dephosphorylation of the MELT motifs on \protein{Knl1} by phosphatases as well as the shedding of the corona, which ceases the assembly of the MCC \cite{PP2A-B56, DyneinStripsCorona} and (2) the disassembly or degradation of existing MCCs \cite{BubR1MitosisTurnover, CCT-MCCDisassembly, Ubiquitylation-MCCDisassembly, UBR5-MCCDisassembly, TRIP13-p31-MAD2} and (3) the attenuation of the affinity between the MCC and the APC/C \cite{APC-SUMO}.

% (2) Due to the facts that mammalian kinetochore may bind to more than one microtubules during the metaphase and that there are two spindle poles from which the kinetochore microtubules may emanate, even if the end-on kinetochore-microtubule attachment has been fully established, it could still be erroneous. These undesired attachment modes are named syntelic and merotelic attachment. In syntelic attachment, both kinetochores on a duplicated chromosome attach to microtubules emanating from the same spindle pole. In merotelic attachment, a single kinetochore form attachment to microtubules emanating from both spindle poles. Cells deploy error-correction mechanisms to correct these erroneous attachment modes.

A more thorough understanding of the SAC may teach us how to prevent or induce its failure and inspire drug discovery to prevent or treat cancers. The revelation of how different molecules orchestrate to build a \Latin{de facto} logic gate from biochemical reactions in the SAC may also spark innovations in protein engineering and biological circuit design.