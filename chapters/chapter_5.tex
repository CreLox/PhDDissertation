\chapter{Conclusion}
\label{chpt:conclusion}

% Through my studies described in previous chapters, I showed that cooperativity manifest at multiple layers of the core SAC signaling cascade in human cells, which contributes to its sensitivity to effectively delay the progression of mitosis in the presence of even just a small number of unattached or laterally attached kinetochores.

% Human \protein{Knl1} possesses 19 putative MELT motifs scattered throughout its middle region. By employing different numbers of MELT motifs in the signaling scaffold in our engineered eSAC activator in \myref{chpt:2}, we showed that there is phenomenological cooperativity when there are multiple MELT motifs within a single scaffold. The most striking case is when there are six MELT motifs (M3-M3) in the signaling scaffold, wherein cells a smaller number of the eSAC activator may invoke a higher SAC signaling acivity compared to cells with a higher number of the eSAC activator. This phenomenon was not apparent when the signaling scaffold incorporated two, three, or four MELT motifs (see \myref{M2_DoseResponse,M3_DoseResponse,M4_DoseResponse}), probably due to the heterogeneity in the sequences of the MELT motifs and their differential affinity to \protein{Bub1}-\protein{Bub3} \cite{MELTActivity}. In light of variable distances between MELT motifs in human \protein{Knl1}, we further proved that such variation in the distance between two MELT motifs (in the range of 135--311 residues) minimally affected the SAC signaling activity. In \myref{ProzoneEffectModel}, we proposed a model reminiscent of the hook effect in an agglutination test to explain our observations.

% MELT motifs phosphorylated by \protein{Mps1} sequentially recruit other SAC proteins like \protein{Bub1}, \protein{BubR1}, and \protein{Mad1} \cite{Ji2017eLife}. Some of the crucial preconditions in our cooperativity model proposed in Chapter 2 include (1) that the competition among a large number of phosphorylated MELT motifs for the limited pool of SAC proteins effectively diminishes freely diffusive SAC proteins in the cytosol and (2) that the co-localization of multiple SAC proteins on the same \protein{Knl1} scaffold boosts the SAC signaling activity. With the help of genome editing, we showed that the number of there was a negative correlation between the number of signaling kinetochores in a prometaphase cell and the number of \protein{Bub1}, \protein{BubR1}, and \protein{Mad1} proteins recruited per signaling kinetochore. This observation was not due to potential variation in the degree of phosphorylation of \protein{Knl1} at signaling kinetochores. We further showed for the first time that recruitment of \protein{BubR1} by \protein{Bub1} \Latin{per se} contributes to the activity of the kinetochore-based SAC signaling. These pieces of experimental evidence unveiled that the foundation of the phenomenological cooperativity observed in \myref{chpt:2} was the competition among a large number of phosphorylated MELT motifs for the limited pool of downstream SAC proteins, which may promote cooperative SAC signaling when they co-localize in close spatial proximity.

%

A more thorough understanding of the SAC may teach us how to prevent or induce its failure and inspire drug discovery to prevent or treat cancers. The revelation of how different molecules orchestrate to build a \Latin{de facto} logic gate from biochemical reactions in the SAC may also spark innovations in protein engineering and biological circuit design.