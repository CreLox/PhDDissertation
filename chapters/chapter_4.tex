\chapter{The structural flexibility of \protein{Mad1} facilitates the coupling between \protein{Mad2}'s conformational change and the formation of the mitotic checkpoint complex (MCC)}
\label{chpt:4}

In previous chapters, we presented evidence on the cooperativity among multiple MELT motifs on the same \protein{Knl1} phosphodomain (the first layer in the SAC signaling cascade). This phenomenon is probably explained by SAC proteins that concurrently bind to these MELT motifs in close spatial proximity (the second layer). In this chapter, we will focus on the spatio-temporal coupling scaffolded by \protein{Mad1} between \protein{Mad2}'s conformational change and the formation of the MCC \cite{BUB1-CDC20-MAD1,Tripartite}. We term it ``the third layer'' in the SAC signaling cascade. % As mentioned in \myref{Chapter3Discussions}, \protein{Mad1} can be recruited to signaling kinetochores by multiple pathway.

The formation of the \protein{Cdc20}-\protein{Mad2} dimer, an MCC subcomplex, is considered to be the rate-limiting step in the assembly of the MCC based on an \Latin{in vitro} study \cite{Faesen2017}. However, how the formation of the \protein{Cdc20}-\protein{Mad2} dimer is catalyzed in the cell was unclear for a long time. Critical studies published later showed that locking cytosolic \protein{Mad2} to the closed conformation inhibited the formation of the \protein{Cdc20}-\protein{Mad2} dimer and compromised the SAC signaling activity \cite{Ma+Poon2016,Ma+Poon2018,Kim2018}. These findings hinted that the conformational change of \protein{Mad2} and the formation of the \protein{Cdc20}-\protein{Mad2} dimer might be temporally coupled. In 2021, two papers published back-to-back (based on either \Latin{in vitro} reconstitution \cite{BUB1-CDC20-MAD1} or studies in \Latin{C. elegans} \cite{Tripartite}) presented direct evidence supporting that \protein{Mad1} scaffolds the spatio-temporal coupling between \protein{Mad2}'s conformational change and the formation of the \protein{Cdc20}-\protein{Mad2} dimer (see \myref{TwoModels}), which offered critical molecular insight into the cooperativity in this process.

In this chapter, we approached the suggested spatio-temporal coupling from a different angle -- by looking into the role of the conserved structural flexibility of \protein{Mad1} in the catalysis of the formation of the \protein{Cdc20}-\protein{Mad2} dimer. We began this hypothesis-driven study independently in 2018, which has evolved into a collaboration project. I am the sole contributor to most cell biology data here. Simon Han performed preliminary tests on the internal tagging of Mad2p in the budding yeast. Results from our collaborators (Dr. Valentina Piano and Dr. Andrea Musacchio from the Max Planck Institute of Molecular Physiology in Germany) will be mentioned to form a coherent story, but those data are not included here.

\section{The ``assembly line model'' \Latin{versus} the ``knitting model''}
\label{TwoModels}

% reasoning: CDC20’s priming to MAD1 C-ter RWD increases its local concentration, which is harnessed only by the knitting model.

\section{The primary sequence of \protein{Mad1}'s loop region is not conserved but the secondary structure is}
The COILS program robustly predicted that amino acids number 582--600 of human \protein{Mad1} do not form coiled coil using various window sizes \cite{LupasCOILS}.

\section{The C-terminus of the \protein{Mad1}-\protein{Mad2} heterotetramer may adopt either a fold-back conformation or an extended conformation both \Latin{in vivo} and \Latin{in vitro}}

\section{\protein{Mad1}'s loop region is important to the SAC signaling activity both \Latin{in vivo} and \Latin{in vitro}}

\section{P585 and P596 are important to the function of \protein{Mad1}'s loop region}

\section{The structural flexibility of \protein{Mad1} enabled by its loop region facilitates the ``knitting'' of the MCC}

\section{Discussions}
\protein{Mad1} has a long half-life under normal conditions \cite{MAD1MAD2Half-life}. And like \protein{Bub1} \cite{Raaijmakers2018, RZZ-MAD1vsBUB1-MAD1_2018, TraceBub1}, even a small pool of \protein{Mad1} (at less than 10\% of its physiological concentration in our knockdown experiments) can maintain a considerable SAC activity in nocodazole-treated cells. Future studies should consider combining RNA interference (or induced knockout of \gene{Mad1}) with acute degradation of \protein{Mad1} proteins to reduce the contribution of the remaining endogenous \protein{Mad1} homodimer to the SAC signaling activity and to minimize the chance of heterodimerization between the remaining endogenous \protein{Mad1} and the rescuing \protein{Mad1} variants in such knockdown-rescue experiments. %(AID, Trim-Away, etc. for such a nuclear protein during the interphase/prophase).

Given the critical role of \protein{Mad1}'s structural flexibility enabled by its loop region, it would be interesting to replace the flexible loop with a turn to lock \protein{Mad1} in the fold-back conformation and see how the SAC signaling activity is affected. Another way to advance our understanding is to investigate whether the two pools of \protein{Mad1} (adopting either the fold-back or extended conformation) inter-convert at a physiologically meaningful rate in the cell using single-molecular approaches. Even though the two proline residues (P585 and P596) in \protein{Mad1}’s loop region are important to the SAC signaling activity, no \protein{Mad1}-interacting protein with either FK506-binding or peptidylprolyl cis-trans isomerase activity has been identified in the PrePPI database using the gene ontology enrichment analysis as of March 2022 \cite{PrePPI}. Additionally, it might be worth finding out whether the equilibrium between the two conformations in the cell is the same as purified \protein{Mad1}-\protein{Mad2} heterotetramer \Latin{in vitro}, which would tell us if the conformation distribution is under active regulation in the cell that costs energy. % However, this does not completely rule out the possibility because the interaction between \protein{Mad1} and a peptidylprolyl isomerase might be transient.

Two missense variants (D587N and A593V) related to \protein{Mad1}’s loop region were recorded in the Genomic Data Commons Data Portal as of March 2022 \cite{GDC}, but the impact of both point mutations is predicted to be benign. Therefore, the physiological impact of potential mutations in \protein{Mad1}’s loop region at the organism level is unclear. It would be interesting to see the physiological impact of introducing point mutations (for example, the multiple proline residues) in \protein{Mad1}’s loop region in various model systems.

In addition to \protein{Cdc20}, closed \protein{Mad2} also interacts with many other proteins (including \protein{Mad1}, \protein{Sgo2} \cite{SGO2-MAD2}, the insulin receptor \cite{MCC_IREndocytosis}, and \protein{Kif20a} \cite{KIF20A-MAD2}), likely by a similar ``safety belt'' mechanism \cite{Structure1GO4}. One question that comes up naturally is whether the same catalytic mechanism (the spatio-temporal coupling between \protein{Mad2}'s conformational change and the formation of the \protein{Cdc20}-\protein{Mad2} dimer) similarly applies to how \protein{Mad2} binds to other proteins (or even more generally, whether the same catalytic mechanism applies to how other HORMA domain proteins bind to other proteins \cite{HORMAReview})? One interesting finding is that the S214A mutation in human \protein{Mad1} impairs the homodimerization of \protein{Mad1} as well as the interaction between \protein{Mad1} and \protein{Mad2} \cite{ATMPhosphorylatesMad1S214}. S214A is unlikely to affect the binding of \protein{Mad2} to \protein{Mad1}'s MIM directly, given the structure of the \protein{Mad1}-\protein{Mad2} heterotetramer \cite{Structure1GO4} and the fact that S214 and MIM are over 300 amino acids apart. This suggested that the homodimerization of \protein{Mad1} might facilitate the binding of \protein{Mad2} to \protein{Mad1}. Future experiments are needed to elucidate the structural and catalytic basis of how \protein{Mad2} ``buckles up'' its binding partners.

\section{Materials and Methods}
For methods of cell culture and Cre-\bacterialgene{lox} RMCE, see \myref{CellCulture+RMCE_Methods}.