\chapter{The structural flexibility of \protein{Mad1} facilitates the coupling between \protein{Mad2}'s conformational change and the formation of the mitotic checkpoint complex (MCC)}
\label{chpt:4}

In previous chapters, we presented evidence on the cooperativity among multiple MELT motifs on the same \protein{Knl1} phosphodomain (the first layer in the SAC signaling cascade). This phenomenon is probably explained by SAC proteins that concurrently bind to these MELT motifs in close spatial proximity (the second layer). In this chapter, we will focus on the spatio-temporal coupling scaffolded by \protein{Mad1} between \protein{Mad2}'s conformational change and the formation of the MCC \cite{BUB1-CDC20-MAD1,Tripartite}. We term it ``the third layer'' in the SAC signaling cascade. % As mentioned in \myref{Chapter3Discussions}, \protein{Mad1} can be recruited to signaling kinetochores by multiple pathway.

The formation of the \protein{Cdc20}-\protein{Mad2} dimer, an MCC subcomplex, is considered to be the rate-limiting step in the assembly of the MCC based on an \Latin{in vitro} study \cite{Faesen2017}. However, how the formation of the \protein{Cdc20}-\protein{Mad2} dimer is catalyzed in the cell was unclear for a long time. Critical studies published later showed that locking cytosolic \protein{Mad2} to the closed conformation inhibited the formation of the \protein{Cdc20}-\protein{Mad2} dimer and compromised the SAC signaling activity \cite{Ma+Poon2016,Ma+Poon2018,Kim2018}. These findings hinted that the conformational change of \protein{Mad2} and the formation of the \protein{Cdc20}-\protein{Mad2} dimer might be temporally coupled. In 2021, two papers published back-to-back (based on either \Latin{in vitro} reconstitution \cite{BUB1-CDC20-MAD1} or studies in \Latin{C. elegans} \cite{Tripartite}) presented direct evidence supporting that \protein{Mad1} scaffolds the spatio-temporal coupling between \protein{Mad2}'s conformational change and the formation of the \protein{Cdc20}-\protein{Mad2} dimer (see \myref{TwoModels}), which offered critical molecular insight into the cooperativity in this process.

In this chapter, we approached the suggested spatio-temporal coupling from a different angle -- by looking into the role of the conserved structural flexibility of \protein{Mad1} in the catalysis of the formation of the \protein{Cdc20}-\protein{Mad2} dimer. We began this hypothesis-driven study independently in 2018, which has evolved into a collaboration project. I am the sole contributor to most cell biology data here. Simon Han performed preliminary tests on the internal tagging of Mad2p in the budding yeast. Results from our collaborators (Dr. Valentina Piano and Dr. Andrea Musacchio from the Max Planck Institute of Molecular Physiology in Germany) will be mentioned to form a coherent story, but those data are not included here.

\section{The ``assembly line model'' \Latin{versus} the ``knitting model''}
\label{TwoModels}

% reasoning: CDC20’s priming to MAD1 C-ter RWD increases its local concentration, which is harnessed only by the knitting model.

\section{The primary sequence of \protein{Mad1}'s loop region is not conserved but the secondary structure is}
The COILS program robustly predicted that amino acids number 582--600 of human \protein{Mad1} do not form coiled coil using various window sizes \cite{LupasCOILS}.

\section{Discussions}


\section{Materials and Methods}
For methods of cell culture and Cre-\bacterialgene{lox} RMCE, see \myref{CellCulture+RMCE_Methods}.