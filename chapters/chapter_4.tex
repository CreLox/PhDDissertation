\chapter{The structural flexibility of MAD1 facilitates the coupling between MAD2's conformational change and the formation of the mitotic checkpoint complex}
\label{chpt:4}
The COILS program robustly predicted that amino acids number 582--600 of human \protein{Mad1} do not form coiled coil using various window sizes \cite{LupasCOILS}.

\section*{Discussions}

An S214A point mutation in human \protein{Mad1} impairs the homodimerization of \protein{Mad1} as well as \protein{Mad1}-\protein{Mad2} interaction at the same time \cite{ATMPhosphorylatesMad1S214}. S214A is unlikely affecting the binding of \protein{Mad2} to \protein{Mad1} MIM directly given the known structure of the domain that includes MIM \cite{Structure1GO4} and the fact that S214 and MIM are more than 300 amino acids apart. This suggest that the homodimerization of \protein{Mad1} may facilitates the priming of \protein{Mad2} onto \protein{Mad1} MIM.


The authors wish to thank the Single Molecule Analysis in Real-Time (SMART) Center at the University of Michigan, seeded by NSF MRI-R2-ID award DBI-0959823 to Nils G. Walter, and J. Damon Hoff for training and technical advice on the ISS Alba v5 time-resolved confocal microscope.