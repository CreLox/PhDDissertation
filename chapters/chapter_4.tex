\chapter{The structural flexibility of \protein{Mad1} facilitates the coupling between \protein{Mad2}'s conformational change and the formation of the mitotic checkpoint complex (MCC)}
\label{chpt:4}

In previous chapters, we presented evidence on the suggested role of cooperativity between multiple MELT motifs (the first layer) and the SAC proteins that bind to these MELT motifs (the second layer) in close spatial proximity. In this chapter, we focus on the third layer \protein{Mad1}, which scaffolds the spatio-temporal coupling between \protein{Mad2}'s conformational change and the formation of the MCC \cite{BUB1-CDC20-MAD1,Tripartite}. % As mentioned in \myref{Chapter3Discussions}, \protein{Mad1} can be recruited to signaling kinetochores by multiple pathway.

We began this hypothesis-driven study in 2018, inspired by papers showing that locking cytosolic \protein{Mad2} to the closed conformation inhibited the formation of the \protein{Cdc20}-\protein{Mad2} dimer and compromised the SAC signaling activity \cite{Ma+Poon2016,Ma+Poon2018,Kim2018}. These findings strongly suggested that the conformational change of \protein{Mad2} and the formation of the MCC is temporally coupled. Two papers published in 2021 (based on either \Latin{in vitro} reconstitution \cite{BUB1-CDC20-MAD1} or studies in \Latin{C. elegans} \cite{Tripartite}) provided more evidence supporting that \protein{Mad1} scaffolds the spatio-temporal coupling between \protein{Mad2}'s conformational change and the formation of the \protein{Cdc20}-\protein{Mad2} dimer. The formation of the \protein{Cdc20}-\protein{Mad2} dimer, an MCC subcomplex, is considered to be the rate-limiting step in the assembly of the MCC based on an \Latin{in vitro} study \cite{Faesen2017}. These two studies provided critical molecular insight into the cooperativity in this process.

In this chapter, we approached the suggested spatio-temporal coupling from a different angle by looking into the role of the conserved structural flexibility of \protein{Mad1} in the catalysis of the formation of the \protein{Cdc20}-\protein{Mad2} dimer. This is a collaboration project and I am the sole contributor of most cell biology data here. Simon Han performed some preliminary tests on the internal tagging of Mad2p in the budding yeast. Results from our collaborators (Dr. Valentina Piano and Dr. Andrea Musacchio from Max Planck Institute of Molecular Physiology in Germany) will be mentioned to form a coherent story, but those data are not included here.

\section{The ``assembly line model'' \Latin{versus} the ``knitting model''}

\section{The primary sequence of \protein{Mad1}'s loop region is not conserved but the secondary structure is}
The COILS program robustly predicted that amino acids number 582--600 of human \protein{Mad1} do not form coiled coil using various window sizes \cite{LupasCOILS}.

\section{Discussions}

An S214A point mutation in human \protein{Mad1} impairs the homodimerization of \protein{Mad1} as well as \protein{Mad1}-\protein{Mad2} interaction at the same time \cite{ATMPhosphorylatesMad1S214}. S214A is unlikely affecting the binding of \protein{Mad2} to \protein{Mad1} MIM directly given the known structure of the domain that includes MIM \cite{Structure1GO4} and the fact that S214 and MIM are more than 300 amino acids apart. This suggest that the homodimerization of \protein{Mad1} may facilitates the priming of \protein{Mad2} onto \protein{Mad1} MIM.

\section{Materials and Methods}