\section{Mitosis, spindle, and kinetochore-spindle microtubule attachment in human cells}

Somatic cell division (mitosis) generates two genetically identical daughter cells from a single parental cell. It follows the interphase (during which nuclear DNA is replicated) and is usually divided into the prophase (chromatin condensation), the prometaphase (after the nuclear envelop breakdown or NEBD and before all chromosomes are aligned at the metaphase plate; see the first two cells on the left in \myref{SACRole}), the metaphase, the anaphase (the segregation of sister chromatids), and the telophase (followed by cytokinesis).

To prepare for the equal distribution of the duplicated chromosomes, a dividing cell forms the spindle which attaches from opposite poles of the cell to all chromosomes through adaptor structures named kinetochores. In an ordinary human somatic cell during mitosis, 46 chromosomes gradually attach to the spindle through the adaptor structure kinetochore during the prometaphase. After all chromosomes are bipolarly attached to the spindle and aligned at the metaphase plate (this process is also named as ``chromosome congression''), the cell will enter the anaphase when sister chromatids separate with each other. The spindle will then pull the two sets of chromosomes towards opposite poles, which later become parts of the two daughter cells.

The architecture of human kinetochores is proposed to resemble a Velcro-like interface for microtubule binding \cite{Velcro}. This ``lawn model'' proposes that the human kinetochore resembles a network of adaptable attachment sites for an average number of about 17 microtubules in the metaphase \cite{Wendell1993, Zaytsev2014, Zaytsev2015, Kukreja2020}. Around the outer kinetochores, a crescent-shaped fibrous structure named the corona is assembled. It is formed through the nucleation of the ROD-Zwilch-ZW10-Spindly (RZZS) complex and is stripped off the kinetochore by dynein at the establishment of the attachment to spindle microtubules.

There are two different modes of attachment in human cells: end-on and lateral. Lateral attachment may convert into end-on attachment. The establishment of the kinetochore-spindle microtubule attachment (henceforth simply referred to as ``the kinetochore-microtubule attachment'') is a gradual and stochastic process. Due to the facts that a human kinetochore binds to an average number of about 17 microtubules during the metaphase of mitosis and that there are two spindle poles from which the spindle microtubules may emanate, even if kinetochores fully established end-on attachment, it could still be erroneous from the perspective of a pair of kinetochores on a duplicated chromosome. These undesired attachment modes are named syntelic and merotelic attachment. In syntelic attachment, both kinetochores on a duplicated chromosome attach to the same spindle pole. In merotelic attachment, a single kinetochore form attachment to microtubules emanating from both spindle poles. Error correction mechanisms are deployed to correct these attachment modes, which are beyond the scope of my thesis.

Corona contribute to chromosome congression by facilitating microtubule capture. The corona also recruits the centromere-associated kinesin \protein{Cenp-E}, which promotes end-on attachment to spindle microtubules. Additionally, another corona protein \protein{Cenp-F} contributes to the stability of the kinetochore-microtubule attachment and maintains the corona by limiting the dynein-mediated stripping of the corona \cite{CENP-FLimitsStripping}.

% \protein{BubR1} binds to the phosphatase \protein{PP2A-B56}, which stabilizes the kinetochore-microtubule attachment.