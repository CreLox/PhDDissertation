\section{\protein{Mad1}'s loop region is important to the SAC signaling activity \Latin{in vivo}}
\label{LoopDeletionSection}

Now that we have shown that the \protein{Mad1}-\protein{Mad2} heterotetramer can assume a fold-back conformation likely enabled by the flexibility provided by the loop region, we next seek to determine whether the loop region is important to the SAC signaling activity. We integrated the expression cassette of either \protein{Mad1}-mNG or \protein{Mad1}\textDelta{}L-mNG under the regulation of a TRE into the genome of HeLa-A12 cells using Cre-\bacterialgene{lox} RMCE (see \myref{Cre-lox}). We then knock down endogenous \protein{Mad1} in these stably-transfected cells using siRNAs that target the 3'-UTR of \gene{Mad1} \cite{siMAD1-3UTR} (which will be simply referred to as si\gene{Mad1}) and induce the expression of \protein{Mad1}(WT/\textDelta{}L)-mNG (si\gene{Mad1}-resistant due to their lack of the endogenous 3'-UTR) by doxycycline. Similar to \myref{per_se}, we used our previous genome-edited \gene{Mad1}-mNG HeLa-A12 cell line as a reference of the endogenous level of \protein{Mad1}. We found out that knocking down \gene{Mad1} using si\gene{Mad1} for two days cripples the SAC signaling activity. However, cells with less than 10\% of its physiological level of the \protein{Mad1} protein still arrest in mitosis for around six hours in \SI{100}{nM} nocodazole, indicating that even a small pool of \protein{Mad1} can maintain a considerable level of SAC signaling activity.

Surprisingly, \protein{Mad1}\textDelta{}L has impaired support for the SAC compared to the wildtype \protein{Mad1} in a dominant-negative manner. 