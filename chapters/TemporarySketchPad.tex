\section{Mitosis, spindle, and the kinetochore-microtubule attachment in human cells}

The architecture of human kinetochores is proposed to resemble a Velcro-like interface for microtubule binding \cite{Velcro}. This ``lawn model'' proposes that the human kinetochore behaves like a network of adaptable attachment sites for an average number of about 17 microtubules in the metaphase \cite{Wendell1993, Zaytsev2014, Zaytsev2015, Kukreja2020}.

% somatic cell division (mitosis)

%  Over the course of prometaphase, the number of unattached kinetochores decreases from 92 to 0 in human cells.

% To prepare for the equal distribution of chromosomes, the dividing cell forms the spindle which attaches from opposite poles of the cell to all chromosomes through adaptor structures named kinetochores. Unattached kinetochores activate the SAC through a series of biochemical reactions.  First, unattached kinetochores recruit the kinase MPS1 which phosphorylates multiple consensus sites named MELT motifs on the scaffold protein KNL1. Phosphorylated MELT motifs will then recruit a series of SAC proteins and eventually generate a diffusible inhibitory signaling molecule known as the Mitotic Checkpoint Complex (MCC) which blocks the progression of cell division. This block allows time for the spindle to find the loose kinetochores. Once all attachment is secured, the SAC will inactivate and the block will be released. The spindle will then pull the two sets of chromosomes towards opposite poles, which later become parts of the two daughter cells.

% different modes of attachment

\section{The SAC signaling in human cells: the core pathway \Latin{versus} the RZZS pathway}

% Figure: KNL1 19 putative MELT motifs & Biochemical event scheme of SAC

The spindle assembly checkpoint (SAC) is a signaling pathway in mitosis that delays anaphase onset in the presence of kinetochores lacking end-on attachment to spindle microtubules \cite{LateralAttachmentSAC}. The SAC can be activated by a single unattached kinetochore, but the strength of the signaling output (as commonly quantified by the duration from the nuclear envelope breakdown to the anaphase onset or mitotic slippage) increases with the number of signaling kinetochores in mammalian cells \cite{RiederNormalProgression,Rheostat,Ablation}. How the number of signaling kinetochores changes the strength of the signaling output is not fully understood. This knowledge will help us comprehend how the SAC prevents premature anaphase onset in the presence of a decreasing number of signaling kinetochores from the nuclear envelope breakdown to the metaphase during normal mitotic progression.

The SAC in mammalian cells relies on the kinetochore-localized signaling scaffold \protein{Knl1} as well as the corona, a crescent-shaped fibrous structure near kinetochores without end-on attachment to spindle microtubules \cite{GSK923295LateralAttachmentEM,CoronaActivatesSAC}. \protein{Knl1} possesses multiple MELT motifs \cite{MELTEvolution}, which are phosphorylated at unattached kinetochores and dephosphorylated at kinetochores with end-on attachment to spindle microtubules. Phosphorylated MELT motifs recruit SAC proteins like \protein{BubR1}, \protein{Bub1}, and \protein{Mad1}, promoting the formation of the mitotic checkpoint complex. The mitotic checkpoint complex inhibits the ubiquitin ligase anaphase promoting complex/cyclosome (APC/C), which ubiquinates the key mitosis regulator Cyclin B1 and targets it to proteasome-mediated degradation. In addition to the \protein{Knl1} pathway, the corona also recruits \protein{Mad1} and contributes to the spindle assembly checkpoint signaling output.

The \protein{Knl1} and the corona machinery also contribute to mitotic progression by facilitating chromosome congression. \protein{Knl1}-recruited \protein{BubR1} binds to the phosphatase \protein{PP2A-B56}, which stabilizes the kinetochore-microtubule attachment. The corona recruits the centromere-associated kinesin \protein{Cenp-E}, which promotes end-on attachment to spindle microtubules. This chromosome congression-promoting function complicates the quantitative study of the SAC signaling output in live cells.

An unattached kinetochore activates the SAC by allowing Mps1 kinase to phosphorylate KNL1 at sites known as MELT motifs due to their consensus sequence (Figures 1B and 1C) [7–10]. This event is followed by the sequential recruitment of Bub3-Bub1 and Mad1-Mad2, along with Bub3-BubR1 and Cdc20 to the kinetochore, with Mps1 phosphorylation playing a licensing role for each step (Figure 1B) [2, 11–19]. We refer to this biochemical cascade as the ‘‘core SAC signaling cascade’’ (dashed gray box in Figure 1B).

In metazoa, the core SAC signaling cascade is complemented by the RZZS pathway, which contributes to the recruitment of Mad1-Mad2 to signaling kinetochores [20].

Ultimately, Bub3, BubR1, Mad2, and Cdc20 form the MCC, which then disperses throughout the cellular volume to inhibit APC/C. The MCC inhibits the Anaphase-Promoting Complex/Cyclosome (APC/C) to delay anaphase onset (Izawa and Pines, 2015; Sudakin et al., 2001). This delay enables the unattached kinetochores to attach to spindle microtubules and thereby ensures accurate chromosome segregation.

The corona assembles on mitotic kinetochores to promote microtubule capture and SAC signaling.

Other constituents of the corona include \protein{Cenp-E} and \protein{Cenp-F}, which interact with \protein{BubR1} and \protein{Bub1}, respectively \cite{CENPELocalization-BUBR1, CENP-FLimitsStripping}.