Switch-like activation of the spindle assembly checkpoint (SAC) is critical for accurate chromosome segregation and for cell division in a timely manner. To determine the mechanisms that achieve this, we engineered a kinetochore-independent ectopic SAC (eSAC) activator, which stimulates the SAC signaling by artificially dimerizing Mps1 kinase domain and a cytosolic KNL1 phosphodomain, the kinetochore signaling scaffold. By exploiting variable eSAC expression in a cell population, we defined the dependence of the eSAC-induced mitotic delay on eSAC concentration in a cell to reveal the dose-response behavior of the core signaling cascade of the SAC. These data uncover two crucial properties of the core SAC signaling cascade: (1) a cellular limit on the maximum anaphase-inhibitory signal that the cascade can generate due to the limited supply of SAC proteins and (2) the ability of the KNL1 phosphodomain to produce the anaphase-inhibitory signal synergistically, when it recruits multiple SAC proteins simultaneously. We propose that these properties together achieve inverse, non-linear scaling between the signal output per kinetochore and the number of signaling kinetochores. When the number of kinetochores is low, synergistic signaling by \protein{Knl1} enables each kinetochore to produce a disproportionately strong signal output. However, when many kinetochores signal concurrently, they compete for a limited supply of SAC proteins, which lowers their signal output. Thus, the signaling activity of unattached kinetochores will adapt to the changing number of signaling kinetochores to enable the SAC to approximate switch- like behavior.



INTRODUCTION
Accurate chromosome segregation during cell division requires that the sister kinetochores on each replicated chromosome are stably attached to microtubules emanating from opposite spindle poles before the cell divides. If one or more kinetochores fail to attach to microtubules, they activate a biochemical signaling cascade known as the spindle assembly checkpoint (SAC). This cascade produces an anaphase-inhibitory signal known as the ‘‘mitotic checkpoint complex’’ (MCC). MCC in- hibits the anaphase promoting complex/cyclosome (APC/C) to prevent the cell from transitioning from prometaphase to anaphase, thus avoiding chromosome mis-segregation [2].

The manner in which the SAC responds to the number of unattached kinetochores in a dividing cell is just as important to its function as its molecular mechanisms. Ideally, it should respond like a switch: it should be ‘‘on’’ if the cell contains one or more unattached kinetochores; otherwise, it should remain ‘‘off’’ (Figure 1A, red curve). This behavior will maximize the accuracy of chromosome segregation and minimize unnecessary delays in anaphase onset. However, realizing this behavior is challenging, because the signaling cascade of the SAC must ensure that (1) one unattached kinetochore, despite its limited signaling capacity, inhibits anaphase onset, but (2) many unat- tached kinetochores present in prophase do not produce a pro- portionately stronger inhibition. Failure to meet one or the other requirement will generate either a weak or an overactive SAC, respectively (gray and black dotted lines in Figure 1A). A single unattached kinetochore delays anaphase onset in Potoroo kid- ney cells and in budding yeast [4, 5]. In human cells, quantitation of APC/C activity as a function of the number of unattached chro- mosomes also reveals a switch-like inhibition of APC/C by one unattached chromosome, implying switch-like SAC activation (data from [3], replotted in Figure 1A, right). Interestingly, when a cell contains many unattached chromosomes, the SAC signaling activity becomes weakly correlated with unattached chromosomes number [6]. The mechanisms underlying this complex SAC behavior in human cells remain unknown.

To understand the basis of the complex behavior of the SAC in human cells, it is necessary to quantify its ‘‘dose-response’’ characteristics by defining how the MCC signal scales with the number of signaling kinetochores. However, obtaining quanti- tative dose-response data is extremely challenging, because this requires the daunting task of generating and then main-
taining specific numbers of unattached kinetochores in a dividing cell [3]. Identifying a specific mechanism mediating switch-like response poses an additional challenge because of the complexity of the SAC (Figure 1B). Current models sug- gest that an unattached kinetochore activates the SAC by allowing Mps1 kinase to phosphorylate KNL1 at sites known as MELT motifs due to their consensus sequence (Figures 1B and 1C) [7–10]. This event is followed by the sequential recruitment of Bub3-Bub1 and Mad1-Mad2, along with Bub3-BubR1 and Cdc20 to the kinetochore, with Mps1 phos- phorylation playing a licensing role for each step (Figure 1B) [2, 11–19]. We refer to this biochemical cascade as the ‘‘core SAC signaling cascade’’ (dashed gray box in Figure 1B). In metazoa, the core SAC signaling cascade is complemented by the RZZ pathway, which independently recruits additional Mad1-Mad2 to the kinetochore [20]. Numerous additional kinases and phosphatases provide additional regulation to these interdependent interactions, but they are not shown here. Ultimately, Bub3, BubR1, Mad2, and Cdc20 form the MCC, which then disperses throughout the cellular volume to inhibit APC/C.

The SAC can be activated independently of kinetochores by overexpressing Mps1 [21], by localizing SAC proteins to metaphase kinetochores [22], or by dimerizing KNL1 phos- phodomain with Mps1 [23–25]. Using these insights, we engi- neered a method to hijack the core SAC signaling cascade from kinetochores in human cells and to analyze SAC signaling quantitatively and with high resolution. We refer to this ectopic SAC activator as the ‘‘eSAC.’’ The eSAC uses controllable heterodimerization of a cytosolic fragment of KNL1, which serves as the scaffold for SAC signaling, and the Mps1 kinase domain, which licenses SAC signaling (Fig- ure 1B). Although the eSAC lacks the full complexity of kinet- ochore-based SAC signaling, it makes the reactions of the core cascade amenable to quantitative analyses dictated by mass-action rates. Using the eSAC system, we uncover two critical design features of the core signaling cascade of the SAC: (1) synergistic signaling by KNL1 and (2) a cellular re- striction on maximal signal generated imposed by the limited abundance of key SAC proteins.