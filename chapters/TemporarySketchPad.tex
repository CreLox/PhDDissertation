\section{Mitosis, spindle, and the kinetochore-microtubule attachment in human cells}

% somatic cell division (mitosis)

The architecture of human kinetochores is proposed to resemble a Velcro-like interface for microtubule binding \cite{Velcro}. This ``lawn model'' proposes that the human kinetochore behaves like a network of adaptable attachment sites for an average number of about 17 microtubules in the metaphase \cite{Wendell1993, Zaytsev2014, Zaytsev2015, Kukreja2020}.

% corona, a crescent-shaped fibrous structure near kinetochores without end-on attachment to spindle microtubules. RZZS

% To prepare for the equal distribution of chromosomes, the dividing cell forms the spindle which attaches from opposite poles of the cell to all chromosomes through adaptor structures named kinetochores. Over the course of prometaphase, the number of unattached kinetochores decreases from 92 to 0 in human cells. The spindle will then pull the two sets of chromosomes towards opposite poles, which later become parts of the two daughter cells.

% different modes of attachment (from the perspective of a single kinetochore: end-on, lateral, unattached; from the perspective of a pair of kinetochores on a duplicated chromosome: monotelic, syntelic, amphitelic, merotelic)

% \protein{Knl1} and the corona also contribute to mitotic progression by facilitating chromosome congression. For example, \protein{BubR1} binds to the phosphatase \protein{PP2A-B56}, which stabilizes the kinetochore-microtubule attachment. The corona recruits the centromere-associated kinesin \protein{Cenp-E}, which promotes end-on attachment to spindle microtubules. The corona promotes microtubule capture.
