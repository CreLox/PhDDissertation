\section{suggests synergy between multiple MELTs on a \protein{Knl1} phosphodomain}

The disproportionately large increase in the signaling strength of the recombinant \protein{Knl1} phosphodomain with multiple MELT motifs indicates synergistic activity. Because if each MELT motif on a \protein{Knl1} phosphodomain generates the MCC independently, then eSAC activators with recombinant phosphodomains containing larger numbers of MELT motifs will simply approach the limit imposed by SAC protein(s) at lower eSAC abundance (Figure 4F). This synergy enables the phosphodomain to generate MCC at a significantly higher rate than that achievable by independent signaling by the MELT motifs. As the concentration of the eSAC activator complex increases, these complexes compete with one another to recruit SAC proteins from the limited pool available in the cell. Individual eSAC activator complexes can no longer recruit multiple SAC proteins, and their synergistic activity weakens. This lead to the gradual decline in the response observed with increasing concentrations of the eSAC activator complex (Figures 4B and 4C).



Design considerations for building the eSAC system
We focused our analysis mainly on a contiguous section within the KNL1 phosphodomain spanning residues 880 to 1020 and containing the 12th, 13th, and 14th MELT motifs [42]. This choice was informed by the following data. The ability of the KNL1 phosphodomain to activate the core SAC signaling cascade is entirely derived from a ∼20 amino acid region spanning each MELT motif [11, 42] (also see Figure S2B). Furthermore, the three motifs are representative of the three classes of MELT motifs based on their ability to recruit Bub3-Bub1 to unattached kinetochores [9].

The spatial distribution of MELT motifs within the eSAC phosphodomains, especially in the case of the eSAC phosphodomain with six MELT motifs was similar to the spatial distribution of MELT motifs in the KNL1 phosphodomain, which is defined by unstructured domains of variable lengths and divergent sequence (see Figure S4G). Finally, previous work demonstrated that this phosphodomain can replace the SAC signaling activity of the entire KNL1 phosphodomain as judged by the metaphase arrest induced by taxol treatment in HeLa cells [9]. Therefore, the artificial eSAC phosphodomain built by two tandem repeats of the selected section of the KNL1 phosphodomain (residues 880-1014) is not expected to possess unique properties that do not exist in KNL1.