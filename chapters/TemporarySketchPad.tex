\section{\protein{Mad1}'s loop region is important to the SAC signaling activity \Latin{in vivo}}
\label{LoopDeletionSection}

We next sought to determine whether the loop region that likely enabled the fold-back conformation of \protein{Mad1} is important to the SAC signaling activity \Latin{in vivo}. We integrated the expression cassette of either \protein{Mad1}-mNG or \protein{Mad1}\textDelta{}L-mNG into the genome of HeLa-A12 cells using Cre-\bacterialgene{lox} RMCE (see \myref{Cre-lox}). We then knocked down endogenous \protein{Mad1} in these stably-transfected cells using siRNAs that target the 3'-UTR of \gene{Mad1} \cite{siMAD1-3UTR} (henceforth collectively referred to as si\gene{Mad1}) and induced the expression of \protein{Mad1}(WT/\textDelta{}L)-mNG (si\gene{Mad1}-resistant due to their lack of the endogenous 3'-UTR) by doxycycline. Our genome-edited \gene{Mad1}-mNG HeLa-A12 cell line served as a reference of the endogenous level of \protein{Mad1} in our live-cell fluorescence imaging.

We found out that knocking down \gene{Mad1} by transfecting the cells with si\gene{Mad1} for two days crippled the SAC signaling activity. However, cells with less than 10\% of the physiological level of \protein{Mad1} on average (estimated from the immunoblot in \myref{MAD1Rescue_WB}) still arrested in mitosis for a considerable amount of time when treated with \SI{100}{nM} nocodazole (only about two hours less than the control group with a physiological level of \protein{Mad1}). Increasing the dosage or times of si\gene{Mad1} treatment or extending the treatment by about half a day did not further decrease the SAC activity, indicating that even a small pool of \protein{Mad1} could sustain a considerable level of the SAC signaling activity.

% figure
% \myref{MAD1Rescue_WB}
% add t-test between knockdown and knockdown + ∆L

Surprisingly, \protein{Mad1}\textDelta{}L had impaired support for the SAC in a dominant-negative manner: cells treated with si\gene{Mad1} and were rescued by a physiological level of \protein{Mad1}\textDelta{}L arrested in mitosis for a significantly shorter duration than cells which were not rescued. As a comparision, wildtype \protein{Mad1}-mNG fully rescued cells treated with si\gene{Mad1}. One possible explanation was that \protein{Mad1}\textDelta{}L-mNG dimerized with the remaining endogenous \protein{Mad1} and restricted its structural flexibility. Structural prediction of the heterodimer between wildtype \protein{Mad1} and \protein{Mad1}\textDelta{}L showed that the loop region of the wildtype \protein{Mad1} introduced a bulge but not enabling the fold-back conformation due to stiffness of the fused \textalpha{}-helix of \protein{Mad1}\textDelta{}L (see \myref{MAD1-MAD1DeltaL_ColabFoldPrediction}). To experimentally confirm this hypothesis, we pulled down doxycycline-induced \protein{Mad1}(wildtype/\textDelta{}L)-mNG from the lysates of HeLa-A12 cells wherein the endogenous \protein{Mad1} was not knocked down. We found that endogenous \protein{Mad1} was also pulled down by both \protein{Mad1}-mNG and \protein{Mad1}\textDelta{}L-mNG, but not by mNeonGreen alone. This proves that the ectopic recombinant \protein{Mad1}\textDelta{}L-mNG can indeed heterodimerize with endogenous \protein{Mad1}.

To confirm that the defect in the SAC signaling activity observed in cells rescued by \protein{Mad1}\textDelta{}L was not due to potential defect in its interaction with \protein{Mad2} or any unexpected effect on the expression of certain SAC proteins, we quantified the localization of \protein{Mad2} at signaling kinetochores by fluorescence imaging and quantified the expression level of other SAC proteins by immunoblotting. Genome-edited \gene{Mad2}$\wedge$mScarlet-I HeLa-A12 cells treated with si\gene{Mad1} and rescued by \protein{Mad1}(wildtype/\textDelta{}L)-mNG showed that no difference in the localization of either \protein{Mad1} or \protein{Mad2} at signaling kinetochores. The cellular abundance of some other SAC proteins like \protein{BubR1}, \protein{Cdc20}, and \protein{Bub3} was also not affected by the rescue experiment (see \myref{MAD1Rescue_WB}).

\Latin{In vitro} reconstitution data (not shown here) from our collaborator also suggested that truncating the loop region of \protein{Mad1} reduced the rate of the formation of \protein{Cdc20}-\protein{Mad2}. Although the results of our knockdown-rescue experiment may have been influenced by the incomplete knockdown of the endogenous \protein{Mad1}, all of these experiments together proved that the loop region of \protein{Mad1} is critical to the SAC signaling activity in its own right.

